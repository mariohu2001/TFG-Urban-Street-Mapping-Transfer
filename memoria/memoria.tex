\documentclass[a4paper,12pt,twoside]{memoir}

% Castellano
\usepackage[spanish,es-tabla]{babel}
\selectlanguage{spanish}
\usepackage[utf8]{inputenc}
\usepackage[T1]{fontenc}
\usepackage{lmodern} % Scalable font
\usepackage{microtype}
\usepackage{placeins}
\usepackage{graphicx}


\RequirePackage{booktabs}
\RequirePackage[table]{xcolor}
\RequirePackage{xtab}
\RequirePackage{multirow}

% Links
\PassOptionsToPackage{hyphens}{url}\usepackage[colorlinks]{hyperref}
\hypersetup{
	allcolors = {red}
}

% Ecuaciones
\usepackage{amsmath}

% Rutas de fichero / paquete
\newcommand{\ruta}[1]{{\sffamily #1}}

% Párrafos
\nonzeroparskip

% Huérfanas y viudas
\widowpenalty100000
\clubpenalty100000

% Imágenes

% Comando para insertar una imagen en un lugar concreto.
% Los parámetros son:
% 1 --> Ruta absoluta/relativa de la figura
% 2 --> Texto a pie de figura
% 3 --> Tamaño en tanto por uno relativo al ancho de página
\usepackage{graphicx}
\newcommand{\imagen}[3]{
	\begin{figure}[!h]
		\centering
		\includegraphics[width=#3\textwidth]{#1}
		\caption{#2}\label{fig:#1}
	\end{figure}
	\FloatBarrier
}

% Comando para insertar una imagen sin posición.
% Los parámetros son:
% 1 --> Ruta absoluta/relativa de la figura
% 2 --> Texto a pie de figura
% 3 --> Tamaño en tanto por uno relativo al ancho de página
\newcommand{\imagenflotante}[3]{
	\begin{figure}
		\centering
		\includegraphics[width=#3\textwidth]{#1}
		\caption{#2}\label{fig:#1}
	\end{figure}
}

% El comando \figura nos permite insertar figuras comodamente, y utilizando
% siempre el mismo formato. Los parametros son:
% 1 --> Porcentaje del ancho de página que ocupará la figura (de 0 a 1)
% 2 --> Fichero de la imagen
% 3 --> Texto a pie de imagen
% 4 --> Etiqueta (label) para referencias
% 5 --> Opciones que queramos pasarle al \includegraphics
% 6 --> Opciones de posicionamiento a pasarle a \begin{figure}
\newcommand{\figuraConPosicion}[6]{%
  \setlength{\anchoFloat}{#1\textwidth}%
  \addtolength{\anchoFloat}{-4\fboxsep}%
  \setlength{\anchoFigura}{\anchoFloat}%
  \begin{figure}[#6]
    \begin{center}%
      \Ovalbox{%
        \begin{minipage}{\anchoFloat}%
          \begin{center}%
            \includegraphics[width=\anchoFigura,#5]{#2}%
            \caption{#3}%
            \label{#4}%
          \end{center}%
        \end{minipage}
      }%
    \end{center}%
  \end{figure}%
}

%
% Comando para incluir imágenes en formato apaisado (sin marco).
\newcommand{\figuraApaisadaSinMarco}[5]{%
  \begin{figure}%
    \begin{center}%
    \includegraphics[angle=90,height=#1\textheight,#5]{#2}%
    \caption{#3}%
    \label{#4}%
    \end{center}%
  \end{figure}%
}
% Para las tablas
\newcommand{\otoprule}{\midrule [\heavyrulewidth]}
%
% Nuevo comando para tablas pequeñas (menos de una página).
\newcommand{\tablaSmall}[5]{%
 \begin{table} [ht]
  \begin{center}
   \rowcolors {2}{gray!35}{}
   \begin{tabular}{#2}
    \toprule
    #4
    \otoprule
    #5
    \bottomrule
   \end{tabular}
   \caption{#1}
   \label{tabla:#3}
  \end{center}
 \end{table}
}

%
% Nuevo comando para tablas pequeñas (menos de una página).
\newcommand{\tablaSmallSinColores}[5]{%
 \begin{table}[H]
  \begin{center}
   \begin{tabular}{#2}
    \toprule
    #4
    \otoprule
    #5
    \bottomrule
   \end{tabular}
   \caption{#1}
   \label{tabla:#3}
  \end{center}
 \end{table}
}

\newcommand{\tablaApaisadaSmall}[5]{%
\begin{landscape}
  \begin{table}
   \begin{center}
    \rowcolors {2}{gray!35}{}
    \begin{tabular}{#2}
     \toprule
     #4
     \otoprule
     #5
     \bottomrule
    \end{tabular}
    \caption{#1}
    \label{tabla:#3}
   \end{center}
  \end{table}
\end{landscape}
}

%
% Nuevo comando para tablas grandes con cabecera y filas alternas coloreadas en gris.
\newcommand{\tabla}[6]{%
  \begin{center}
    \tablefirsthead{
      \toprule
      #5
      \otoprule
    }
    \tablehead{
      \multicolumn{#3}{l}{\small\sl continúa desde la página anterior}\\
      \toprule
      #5
      \otoprule
    }
    \tabletail{
      \hline
      \multicolumn{#3}{r}{\small\sl continúa en la página siguiente}\\
    }
    \tablelasttail{
      \hline
    }
    \bottomcaption{#1}
    \rowcolors {2}{gray!35}{}
    \begin{xtabular}{#2}
      #6
      \bottomrule
    \end{xtabular}
    \label{tabla:#4}
  \end{center}
}

%
% Nuevo comando para tablas grandes con cabecera.
\newcommand{\tablaSinColores}[6]{%
  \begin{center}
    \tablefirsthead{
      \toprule
      #5
      \otoprule
    }
    \tablehead{
      \multicolumn{#3}{l}{\small\sl continúa desde la página anterior}\\
      \toprule
      #5
      \otoprule
    }
    \tabletail{
      \hline
      \multicolumn{#3}{r}{\small\sl continúa en la página siguiente}\\
    }
    \tablelasttail{
      \hline
    }
    \bottomcaption{#1}
    \begin{xtabular}{#2}
      #6
      \bottomrule
    \end{xtabular}
    \label{tabla:#4}
  \end{center}
}

%
% Nuevo comando para tablas grandes sin cabecera.
\newcommand{\tablaSinCabecera}[5]{%
  \begin{center}
    \tablefirsthead{
      \toprule
    }
    \tablehead{
      \multicolumn{#3}{l}{\small\sl continúa desde la página anterior}\\
      \hline
    }
    \tabletail{
      \hline
      \multicolumn{#3}{r}{\small\sl continúa en la página siguiente}\\
    }
    \tablelasttail{
      \hline
    }
    \bottomcaption{#1}
  \begin{xtabular}{#2}
    #5
   \bottomrule
  \end{xtabular}
  \label{tabla:#4}
  \end{center}
}



\definecolor{cgoLight}{HTML}{EEEEEE}
\definecolor{cgoExtralight}{HTML}{FFFFFF}

%
% Nuevo comando para tablas grandes sin cabecera.
\newcommand{\tablaSinCabeceraConBandas}[5]{%
  \begin{center}
    \tablefirsthead{
      \toprule
    }
    \tablehead{
      \multicolumn{#3}{l}{\small\sl continúa desde la página anterior}\\
      \hline
    }
    \tabletail{
      \hline
      \multicolumn{#3}{r}{\small\sl continúa en la página siguiente}\\
    }
    \tablelasttail{
      \hline
    }
    \bottomcaption{#1}
    \rowcolors[]{1}{cgoExtralight}{cgoLight}

  \begin{xtabular}{#2}
    #5
   \bottomrule
  \end{xtabular}
  \label{tabla:#4}
  \end{center}
}



\graphicspath{ {./img/} }

% Capítulos
\chapterstyle{bianchi}
\newcommand{\capitulo}[2]{
	\setcounter{chapter}{#1}
	\setcounter{section}{0}
	\setcounter{figure}{0}
	\setcounter{table}{0}
	\chapter*{#2}
	\addcontentsline{toc}{chapter}{#2}
	\markboth{#2}{#2}
}

% Apéndices
\renewcommand{\appendixname}{Apéndice}
\renewcommand*\cftappendixname{\appendixname}

\newcommand{\apendice}[1]{
	%\renewcommand{\thechapter}{A}
	\chapter{#1}
}

\renewcommand*\cftappendixname{\appendixname\ }

% Formato de portada
\makeatletter
\usepackage{xcolor}
\newcommand{\tutor}[1]{\def\@tutor{#1}}
\newcommand{\course}[1]{\def\@course{#1}}
\definecolor{cpardoBox}{HTML}{E6E6FF}
\def\maketitle{
  \null
  \thispagestyle{empty}
  % Cabecera ----------------
\noindent\includegraphics[width=\textwidth]{cabecera}\vspace{1cm}%
  \vfill
  % Título proyecto y escudo informática ----------------
  \colorbox{cpardoBox}{%
    \begin{minipage}{.8\textwidth}
      \vspace{.5cm}\Large
      \begin{center}
      \textbf{TFG del Grado en Ingeniería Informática}\vspace{.6cm}\\
      \textbf{\LARGE\@title{}}
      \end{center}
      \vspace{.2cm}
    \end{minipage}

  }%
  \hfill\begin{minipage}{.20\textwidth}
    \includegraphics[width=\textwidth]{escudoInfor}
  \end{minipage}
  \vfill
  % Datos de alumno, curso y tutores ------------------
  \begin{center}%
  {%
    \noindent\LARGE
    Presentado por \@author{}\\ 
    en Universidad de Burgos --- \@date{}\\
    Tutor: \@tutor{}\\
  }%
  \end{center}%
  \null
  \cleardoublepage
  }
\makeatother

\newcommand{\nombre}{Mario Hurtado Ubierna} %%% cambio de comando

% Datos de portada
\title{Urban Street Mapping Transfer}
\author{\nombre}
\tutor{Virginia Ahedo García y Jesús Manuel Maudes Raedo}
\date{\today}

\begin{document}

\maketitle


\newpage\null\thispagestyle{empty}\newpage


%%%%%%%%%%%%%%%%%%%%%%%%%%%%%%%%%%%%%%%%%%%%%%%%%%%%%%%%%%%%%%%%%%%%%%%%%%%%%%%%%%%%%%%%
\thispagestyle{empty}


\noindent\includegraphics[width=\textwidth]{cabecera}\vspace{1cm}

\noindent Dña. Virginia Ahedo, profesora del departamento de Ingeniería de Organización, área de Organización de Empresas y D.
Jesús Manuel Maudes, profesor del departamento de Ingeniería Informática, área de Lenguajes y Sistemas Informáticos.

\noindent Expone:

\noindent Que el alumno D. \nombre, con DNI 71365182T, ha realizado el Trabajo final de Grado en Ingeniería Informática titulado <<Urban Street Mapping Transfer>>. 

\noindent Y que dicho trabajo ha sido realizado por el alumno bajo la dirección del que suscribe, en virtud de lo cual se autoriza su presentación y defensa.

\begin{center} %\large
En Burgos, {\large \today}
\end{center}

\vfill\vfill\vfill

% Author and supervisor
\begin{minipage}{0.45\textwidth}
\begin{flushleft} %\large
Vº. Bº. del Tutor:\\[2cm]
Dña. Virginia Ahedo
\end{flushleft}
\end{minipage}
\hfill
\begin{minipage}{0.45\textwidth}
\begin{flushleft} %\large
Vº. Bº. del co-tutor:\\[2cm]
D. Jesús Manuel Maudes
\end{flushleft}
\end{minipage}
\hfill

\vfill

% para casos con solo un tutor comentar lo anterior
% y descomentar lo siguiente
%Vº. Bº. del Tutor:\\[2cm]
%D. nombre tutor


\newpage\null\thispagestyle{empty}\newpage




\frontmatter

% Abstract en castellano
\renewcommand*\abstractname{Resumen}
\begin{abstract}
Uno de los principales factores que considerar cuando se busca abrir un nuevo negocio es la localización. Esto plantea un problema conocido como el \textit{Location Problem}, el cual busca encontrar la ubicación más adecuada para una determinada categoría comercial pueda reportar más beneficios. En el caso del proyecto se aplicará al \textit{retailing}, es decir, a tiendas minoristas.

Existen distintas aproximaciones a este problema, en este proyecto se aplicará un enfoque desde la ciencia de redes, teniendo en consideración las distintas categorías comerciales y otros servicios en una serie de ciudades con el objetivo de crear un sistema de recomendación con capacidad de ofrecer propuestas de comercios, así como de ubicaciones dada una categoría comercial en concreto. Para ello se calcularán las relaciones de atracción y repulsión entre las distintas categorías teniendo en cuenta la distribución geográfica de estas.

Para lograr lo anterior se utilizarán datos extraídos de una API de geolocalización de un proyecto conocido como \textit{OpenStreetMap} de uso libre, así como,\textit{Neo4j}, que forma parte de un nuevo paradigma conocido como la orientación a grafos en el contexto de las bases de datos.

Además de contemplar recomendaciones a nivel local, también se empleará la transferencia de información entre ciudades, permitiendo utilizar la información de una ciudad sobre otra diferente, evitando de esta forma un nuevo proceso de extracción de información para esa segunda ciudad. Con este objetivo se creará también un modelo de inteligencia artificial que integre los datos de todas las ciudades.

Para facilitar el uso de la herramienta de este proyecto, se desarrollará una aplicación web sencilla que permita a los usuarios obtener recomendaciones basadas la ubicación o en la categoría comercial.

\end{abstract}

\renewcommand*\abstractname{Descriptores}
\begin{abstract}
%Palabras separadas por comas que identifiquen el contenido del proyecto Ej: servidor web, buscador de vuelos, android \ldots
Ciencia de redes, sistema de recomendación, comercio minorista, \textit{machine learning}, web, transferencia de conocimiento, API de geolocalización, base de datos de grafos.
\end{abstract}

\clearpage

% Abstract en inglés
\renewcommand*\abstractname{Abstract}
\begin{abstract}
One of the main factors to consider when looking to open a new business is location. This poses a problem known as the \textit{Location Problem}, which seeks to find the most suitable location for a given business category that can bring the most profit. In the case of this project, it will be applied to retailing, i.e. retail stores.

There are different approaches to this problem, in this project a network science approach will be applied, taking into consideration the different commercial categories and other services in a series of cities with the objective of creating a recommendation system with the capacity to offer proposals of stores, as well as locations given a specific commercial category. For this purpose, the attraction and repulsion relationships between the different categories will be calculated, taking into account their geographical distribution.

To achieve the above, data extracted from a geolocation API of a project known as OpenStreetMap will be used, as well as Neo4j, which is part of a new paradigm known as graph orientation in the context of databases.

In addition to contemplating recommendations at the local level, the transfer of information between cities will also be used, allowing the use of information from one city on a different one, thus avoiding a new information extraction process for that second city. For this purpose, an artificial intelligence model that integrates data from all cities will also be created.

To facilitate the use of the tool of this project, a simple web application will be developed to allow users to obtain recommendations based on location or commercial category.

\end{abstract}

\renewcommand*\abstractname{Keywords}
\begin{abstract}
Network science, recommendation system, retailing, machine learning, web, knowledge transfer, geolocation API, graph database
\end{abstract}

\clearpage

% Indices
\tableofcontents

\clearpage

\listoffigures

\clearpage

\listoftables
\clearpage

\mainmatter
\capitulo{1}{Introducción}

%Descripción del contenido del trabajo y del estrucutra de la memoria y del resto de materiales entregados.

El conocido como \textit{Retail Location Problem} consiste en un problema al que se enfrentan los negocios a la hora de elegir una ubicación donde abrir un nuevo establecimiento. Un correcto emplazamiento puede incrementar enormemente el rendimiento del negocio, proporcionando ventajas que la competencia difícilmente puede imitar. Esto puede suponer la diferencia entre el éxito o el fracaso 

\textit{Urban Street Mapping Transfer} se trata de un de un proyecto que busca ofrecer soluciones para este problema. La técnica que se utilizará para ello consiste en el análisis de la estructura comercial de cada ciudad mediante el uso de conceptos propios de la ciencia de redes así como de distintos métodos matemáticos y estadísticos que nos permitan estimar qué ubicaciones pueden ser más provechosas para un negocio. También se emplearán modelos de inteligencia artificial para integrar los resultados del análisis en un sistema de recomendación.

Además de lo anterior se utilizará \textit{transfer learning}, una técnica propia del \textit{machine learning} que permitirá utilizar datos de unas ciudades para poder realizar recomendaciones sobre otras.

\capitulo{2}{Objetivos del proyecto}

%Este apartado explica de forma precisa y concisa cuales son los objetivos que se persiguen con la realización del proyecto. Se puede distinguir entre los objetivos marcados por los requisitos del software a construir y los objetivos de carácter técnico que plantea a la hora de llevar a la práctica el proyecto.

Este proyecto tiene como objetivo principal la creación de recomendaciones de categorías comerciales dadas ubicaciones, de forma que se pueda estimar el mejor lugar para como localización de un determinado tipo de establecimiento comercial. Para ello se utilizarán distintos índices de calidad propuestos para este problema así como su uso combinado mediante un modelo de inteligencia artificial destinado a problemas de clasificación. Para facilitar el uso de la herramienta se diseñará una aplicación web sencilla que permita hacer las recomendaciones a un usuario.

Los objetivos para llevar a cabo el proyecto son los siguientes:

\begin{enumerate}
	\item Obtención de establecimientos comerciales con sus respectivas categorías de distintas ciudades mediante el uso de una API de geolocalización.
	\item Uso de una base de datos orientada a grafos donde almacenar las ubicaciones comerciales de forma que podamos operar sobre ella como se haría en una red.
	\item Análisis de la estructura comercial de cada ciudad así como de la interacción entre las distintas categorías.
	\item Uso y cálculo de distintos indices de calidad con objetivo de realizar recomendaciones.
	\item Uso de transferencia de información entre la estructura comercial de distintas ciudades.
	\item Creación de un sistema de recomendación basado en un modelo de inteligencia artificial.
	\item Creación de una aplicación web con objetivo de facilitar el uso de la herramienta.
\end{enumerate}


\capitulo{3}{Conceptos teóricos}

La cuestión que se busca abordar con este proyecto es el \textit{Retail Location Problem}. Este consiste
en la elección de la ubicación en la que colocar una nueva tienda de forma que esta le aporte el máximo beneficio posible. Para abordar esto se emplearán técnicas propias de la ciencia de redes, el aprendizaje automático y el conocido como \textit{knowledge transfer}.

\section{\textit{Retail Location Problem}}

El conocido como <<Retail Location Problem>> trata de un problema enfrentado por las empresas pertenecientes al comercio minorista al elegir una ubicación donde empezar un nuevo negocio. La ubicación, lejos de ser un factor carente de importancia puede marcar la diferencia entre el éxito o el fracaso del comercio. Un negocio adecuadamente posicionado cuenta con ventajas sobre la competencia que pueden hacer aumentar significativamente su rendimiento sobre el resto ~\cite{Ahedo2021, RSVAJSSHJG}.

La elección de la ubicación puede hacerse ateniendo a distintos factores, logísticos, poblacionales, fiscales, entre otros.

 En este proyecto, se emitirán recomendaciones basándose en el ecosistema que rodea a una determinada ubicación, teniendo en cuenta cómo complementan los distintos comercios cercanos al hipotético nuevo negocio. Todo ello se abordará desde el enfoque de Teoría de Redes~\cite{eswiki:151888116}, modelando los ecosistemas comerciales como redes complejas, y analizándolas con las distintas herramientas pertinentes. 

%%consiste en encontrar
%%las mejores ubicaciones para un comercio de determinada categoría de forma 
%%que esta aporte ventajas competitivas con respecto a la competencia.
%
%Originalmente este problema no se limitaba a comercios minoristas como se hará en este
%trabajo, sino que también abarcaba ámbitos como la industria, teniendo en consideración
%no solo la ubicación sino que también la distribución y recursos.
%Se ha intentado buscar solución a este problema utilizando distintos ámbitos como la estadística y las matemáticas


\section{Teoría de grafos}

Los grafos son abstracciones matemáticas compuestas por dos elementos ~\cite{enwiki:1171835383}:

\begin{itemize}
	\item $V$: Vértices o nodos. Suponen la entidad más básica del grafo, pudiendo representar distintos conceptos.
	\item $E$: Enlaces o aristas. Representan las relaciones entre los distintos nodos de un grafo.
\end{itemize}

Dada su naturaleza, los grafos han sido usados habitualmente para modelar relaciones entre elementos.

Los nodos, además de constituir el elemento más básico, pueden presentar atributos,los cuales indican distintas casuísticas, tales como, por ejemplo, la pertenencia a un determinado grupo.

Existen distintos tipos en función de como se modelen las relaciones entre objetos. Si las relaciones no cuentan con dirección alguna se les conoce como no dirigidos, si existe alguna restricción en las direcciones con las que se recorren los enlaces entonces se trata de un grafo dirigido; y en caso de permitir múltiples enlaces entre cada par de nodos se trataría de un multigrafo. También pueden presentar una ponderación en los enlaces entre nodos, siendo de esta forma un grafo pesado~\cite{enwiki:1171835383}.

\subsection{Modelos nulos}

En el ámbito de las matemáticas y la teoría de grafos un modelo nulo se trata de un modelo con una generación aleatoria en alguno de sus aspectos~\cite{enwiki:1169838323}. Los modelos nulos sirven como referencia, ya que nos permiten conocer cuáles serían los patrones que encontraríamos si el fenómeno bajo consideración se hubiera generado de manera aleatoria, al azar. De esta forma se puede extraer ciertas características distintivas de los grafos sujetos a análisis.

En los artículos científicos en los que se inspira este trabajo solo se consideraban las relaciones estadísticamente significativas entre categorías (0.05 de significación). No obstante, en este proyecto se han considerado todas las relaciones, tanto las estadísticamente significativas como las no significativas. Esta decisión fue tomada con el propósito de comprobar si el utilizar toda la información disponible mejoraba el rendimiento de los modelos de predicción. En este sentido, nótese que las diferencias entre los valores del MRR obtenidos en este proyecto y los que obtenían en el artículo de \textit{Knowledge Transfer in Commercial Feature Extraction for the Retail Store Location Problem}~\cite{Ahedo2021} cabe esperar que se deriven precisamente del cambio de aproximación, i.e., de utilizar todos los datos en lugar de solo las relaciones estadísticamente significativas.

\section{Ciencia de Redes}

La ciencia de redes se trata de un campo de investigación multidisciplinar que bebe de distintos ámbitos tales como las matemáticas, la estadística, la minería de datos, entre otros. Utiliza los conocimientos que la teoría de grafos provee para representar elementos y relaciones de la realidad y de esta forma ser capaces de estudiar y analizar sistemas complejos. Los grafos utilizados por esta rama de conocimiento se suelen conocer como redes~\cite{eswiki:151888116}.

Además de basarse en los conceptos de la teoría de grafos, también aplica técnicas propias de la informática y el <<Machine Learning>> para facilitar las labores de análisis de redes. Algunos ejemplos de aplicación de estas técnicas podrían ser la predicción de posibles enlaces entre nodos, detección de comunidades o la clasificación de nodos en distintas categorías.

En los últimos años ha obtenido gran relevancia al aplicarse a ámbitos como las redes sociales, de transporte, de transmisión de enfermedades, entre otros. En el caso de este trabajo, se aplicará para modelar la estructura comercial de las distintas ciudades con el objetivo de hacer recomendaciones en base a la ubicación para determinados negocios.


\section{Técnicas}
Con el objetivo de extraer y analizar la estructura comercial de una ciudad, en la literatura se han propuesto distintos métodos basados en Teoría de Redes, los cuales se diferencian en sus hipótesis subyacentes. En última instancia, los distintos enfoques tienen distintos modelos nulos contra los que chequean la significación estadística de las interacciones comerciales (tanto atractivas como repulsivas). Nótese, que cada método nos permite posteriormente calcular índices de calidad para cada ubicación, los cuales nos dan una idea de la idoneidad de la localización para cada categoría comercial ~\cite{Ahedo2021,RSVAJSSHJG}. Estos nos proveen coeficientes entre las distintas categorías que nos permitirán extraer las relaciones de afinidad y repulsión; además de posteriormente permitirnos los cálculos de los índices de calidad para cada ubicación, pudiendo realizar con ellos las recomendaciones que buscamos en este trabajo.

Se trata de los siguientes:
\begin{itemize}
	\item Jensen
	\item \textit{Permutation}
	\item \textit{Rewiring}
\end{itemize}


Antes de poder aplicar estos métodos será necesario contar con una representación en forma de red de la estructura comercial de la ciudad ~\cite{Ahedo2021,RSVAJSSHJG}. Los nodos de este grafo representarán cada una de las ubicaciones comerciales. En cuanto a las relaciones, estas se crearan en base a la proximidad espacial entre nodos. Para esto se tomará una distancia de 100 metros como máximo para la creación de un enlace i.e., se creará un enlace entre dos nodos (locales comerciales) si se encuentran a una distancia menor o igual que 100 m. (Distancia geodésica).

A continuación, se hará una breve descripción de cada uno de estos métodos. Para ello, utilizaremos la siguiente nomenclatura~\cite{Ahedo2021,RSVAJSSHJG}.
\begin{itemize}
	\item $T$: Conjunto de todas las distintas ubicaciones.
	\item $A,B$: Conjuntos de ubicaciones de determinadas categorías.
	\item \text{null\_model}: Obtenido a partir de las simulaciones de Monte Carlo de \textit{Permutation} y \textit{Rewiring}.
	\item $N_s(p,r)$: Tamaño del conjunto de ubicaciones de categoría $s$ en una proximidad de $r$ metros.
\end{itemize}

\subsection{Jensen}


Jensen propone unos coeficientes de interacción entre categorías ~\cite{Jensen2006}. Este método puede obtener tanto las relaciones de atracción como de repulsión entre categorías comerciales. Las primeras serán aquellas con un valor por encima de 1, y las segundas aquellas en las que esté por debajo de 1.

El cálculo de estos coeficientes es distinto cuando se trata de una categoría sobre sí misma con respecto a una categoría sobre otra distinta.

\begin{equation*}
	\text{Intracategoría: } M_\text{AA} = \frac{|T| - 1}{|A|(|A|-1)} \sum_{a \in A}\frac{N_A(a,r)}{N_T(a,r)}
\end{equation*}

\begin{equation*}
	\text{Intercategoría: }M_\text{AB} = \frac{|T| - |A|}{|A||B|} \sum_{a \in A}\frac{N_B(a,r)}{N_T(a,r) - N_A(a,r)}
\end{equation*}

El cálculo de estos coeficientes llevará a la obtención de una matriz de interacciones que podrá ser utilizada para posteriores cálculos ~\cite{Jensen2006}.

El método de Jensen presenta algunos problemas:

\begin{itemize}
	\item Tiendas aisladas: Si alguna tienda se encuentra aislada del resto esto conlleva a una indeterminación $\frac{0}{0}$ en el cálculo de los coeficientes.
	\item Función logarítmica: A la hora de aplicar los coeficientes de Jensen se emplea su valor logarítmico, haciendo que se comporte de forma asimétrica para valores positivos y negativos. Además, en caso de que el coeficiente sea 0 hace que su valor sea $-\infty$.
\end{itemize}

Dados estos problemas se proponen otros 2 métodos alternativos al de Jensen.
\subsection{Permutation}

El método \textit{Permutation} requiere de la aplicación de simulaciones de Monte Carlo ~\cite{Montecarlo} con modelos nulos sobre la red ~\cite{Ahedo2021,RSVAJSSHJG}. En este caso se crearán redes aleatorias, pero con la restricción de que se mantendrá la estructura real de la red, permutándose aleatoriamente las categorías asociadas a cada nodo. La hipótesis subyacente es la siguiente: se mantiene la estructural comercial global (la ubicación de los emplazamientos comerciales) pero no se mantiene el ecosistema local, ya que se permutan las categorías comerciales.

De cada simulación se obtendrá el valor de interacción entre categorías, para posteriormente obtener la media y la desviación típica de los resultados de las simulaciones ~\cite{Ahedo2021,RSVAJSSHJG}.

Con ello se podrá obtener el Z-Score para cada par de categorías, con un funcionamiento análogo a los coeficiente de Jensen. Se obtendrá una matriz simétrica con ellos.

\begin{equation*}
	Z_{AB} = \frac{x_{AB} - \overline{x_{AB}^{null\_model}}}{s_{AB}^{null\_model}}
\end{equation*}


\imagen{Permutation.drawio}{Ejemplo del método \textit{Permutation}}{1}

\subsection{Rewiring}

\textit{Rewiring} se trata de un método similar al anterior, también está basado en la generación de modelos nulos con simulaciones de Monte Carlo ~\cite{Montecarlo}. La diferencia entre este y el anterior está en las restricciones. En este la categoría comercial es mantenida en cada uno de los nodos, mientras que los enlaces entre nodos son aleatorizados, con la única restricción de que se debe mantener el grado que los nodos poseían en la red original ~\cite{Ahedo2021,RSVAJSSHJG}. La hipótesis bajo Rewiring es que al mantener el grado, estamos preservando la estructura local que se genera en torno a un determinado comercio de una determinada categoría.

Al igual que en el anterior se obtendrá una matriz con los Z-Score por pares de categorías.


\imagen{Rewiring.drawio}{Ejemplo del método \textit{Rewiring}}{1}

\section{Índices de calidad}

Con los coeficientes de interacción entre categorías de la anterior sección se pueden obtener los llamados \textit{quality indices} o índices de calidad ~\cite{Ahedo2021,Jensen2006,RSVAJSSHJG}. Dichos índices cuantifican numéricamente la idoneidad de una localización para una determinada categoría comercial. En el caso de Jensen se aplica el valor del logaritmo del coeficiente, es decir, $a_{ij} = log(M_{ij})$, mientras que en el caso de \textit{Permutation} y \textit{Rewiring} se utilizará el propio Z-Score para esas categorías, $a_{ij} = Z_{ij}$~\cite{Ahedo2021}. 

\begin{equation*}
	Q_i(x,y) \equiv \sum_\textit{j=1}^N a_{ij} (nei_{ij}(x,y) - \overline{nei_{ij}})
\end{equation*}



\begin{itemize}
	\item $a_{ij}$: Interacción de la matriz de coeficientes entre categorías \textit{i} y \textit{j}.
	\item $N$: Número total de categorías.
	\item $nei_{ij}(x,y)$: Número de negocios vecinos con categoría $j$ en torno a ubicación $(x,y)$.
	\item $\overline{nei_{ij}}$: Promedio de vecinos de la categoría $j$ que tienen los nodos de categoría $i$.
\end{itemize}

Con esto se obtendría la adecuación de una ubicación $(x,y)$ con la categoría comercial $i$. Si se parte de un conjunto predefinido de categorías comerciales se puede encontrar el negocio más adecuado para una ubicación $(x,y)$ calculando los índices de calidad para todas las categorías, siendo el más apropiado aquel con mayor valor de $Q_i$ ~\cite{Ahedo2021,Jensen2006}.

\subsection{Índices \textit{Raw}}

La idea que subyace a los índices de calidad de la sección anterior es la de que una ubicación que se asemeje a la ubicación promedio de todas las tiendas minoristas de esa categoría dentro de la ciudad, puede ser un buen local para otra tienda de esa categoría ~\cite{Jensen2006}. Lo anterior se manifiesta al sustraer el valor promedio para cada categoría $\overline{nei_{ij}}$ en el cálculo de los índices de calidad. Alternativamente se proponen unos nuevos que no tienen en consideración los valores medios~\cite{Ahedo2021}. Su cálculo es el siguiente:

\begin{equation*}
	Q_i(x,y) \equiv \sum_\textit{j=1}^N a_{ij}  (nei_{ij}(x,y))
\end{equation*}

\subsection{\textit{Mean Reciprocal Rank}}

Los índices de calidad nos proporcionan una serie de categorías comerciales ordenadas en función de la adecuación de cada una. Para poder evaluar la eficiencia de cada índice se emplea la medida conocida como \textit{Mean Reciprocal Rank}~\cite{MRR}. Esta se suele emplear en sistemas de recomendación para evaluar su rendimiento usando la posición que ocupa la categoría real sobre el \textit{ranking} provisto, en este caso, por los índices de calidad~\cite{Ahedo2021}. Sigue la siguiente formula:

\begin{equation*}
	MRR = \frac{1}{|Q|}  \sum_{i=1}^{|Q|} \frac{1}{rank_i}
\end{equation*}

\begin{itemize}
	\item $Q$: Conjunto de ubicaciones con categoría conocida.
	\item $rank_i$: Posición que ocupa la categoría real de la ubicación en el \textit{ranking} ofrecido por un índice de calidad.
\end{itemize}


\subsection{Uso combinado de índices de calidad}

Si bien el uso simple de los índices de calidad ya es suficiente para obtener recomendaciones, puede darse la situación en que estas difieran dependiendo del método utilizado para obtener los índices. Si bien podemos ver qué método es más eficaz utilizando el MRR y decantarnos por uno o por otro, puede darse el caso de que difieran por estar capturando información diferente. Por este motivo, puede ser interesante hacer uso de todos ellos conjuntamente, agregándolos mediante un modelo de \textit{machine learning}, para que este pueda explotar todos los aspectos complementarios de los mismos.~\cite{Ahedo2021}.

Para poder entrenar el modelo se debe contar primero con un conjunto de datos. Este lo podremos construir con una serie de ubicaciones de las que conozcamos su categoría real, siendo esta la etiqueta a predecir. Los predictores los constituirán los valores de los índices de calidad de cada ubicación obtenidos para cada método y categoría.

El modelo entrenado podrá realizar las funciones de sistema de recomendación, pudiendo a su vez ser evaluado usando el MRR.

En este trabajo se empleará un \textit{Random Forest} ~\cite{RF}, clasificador de la familia de los \textit{ensembles}. El uso de un modelo de esta familia es debido a las ventajas que estos poseen frente a los modelos simples, ofreciendo mejor rendimiento y versatilidad ~\cite{HundredClass}. Estos además presentan un buen equilibrio entre \textit{bias} y varianza, puesto que con sus múltiples árboles es capaz de reducir la varianza mientras que mantiene su capacidad de detectar patrones complejos en los datos manteniendo un \textit{bias} bajo.

\section{\textit{Knowledge Transfer}}

El conocido como \textit{Transfer Learning} ~\cite{TRANSFERLEARNING} se trata de una técnica propia del aprendizaje automático que consiste en utilizar conocimiento obtenido previamente para resolver nuevos problemas similares. Esto permite mejorar el rendimiento ya que se cuentan con datos previos que pueden ser reutilizados.

En el caso de este trabajo el \textit{Knowledge Transfer} aplicará calculando los índices de calidad de una ciudad utilizando la matriz de interacciones de los distintos métodos de otra ~\cite{Ahedo2021}. A su vez estos pueden ser utilizados de forma combinada mediante un modelo de inteligencia artificial.


La aplicación de esta técnica puede ser de gran importancia ya que permitiría realizar predicciones sobre nuevas ciudades utilizando datos con los que ya se contaba anteriormente. Además permitiría analizar si las distintas ciudades comparten una estructura comercial similar, pudiendo utilizarse con bastante confianza para predecir sobre nuevas; o si, por el contrario cada ciudad posee una estructura distinta y más específica con respecto a otras.

Al igual que con las predicciones que se realizaban a nivel local, estas podrán ser evaluadas utilizando el MRR.
	
% En aquellos proyectos que necesiten para su comprensión y desarrollo de unos conceptos teóricos de una determinada materia o de un determinado dominio de conocimiento, debe existir un apartado que sintetice dichos conceptos.

% Algunos conceptos teóricos de \LaTeX \footnote{Créditos a los proyectos de Álvaro López Cantero: Configurador de Presupuestos y Roberto Izquierdo Amo: PLQuiz}.

% \section{Secciones}

% Las secciones se incluyen con el comando section.

% \subsection{Subsecciones}

% Además de secciones tenemos subsecciones.

% \subsubsection{Subsubsecciones}

% Y subsecciones. 


% \section{Referencias}

% Las referencias se incluyen en el texto usando cite \cite{wiki:latex}. Para citar webs, artículos o libros \cite{koza92}.


% \section{Imágenes}

% Se pueden incluir imágenes con los comandos standard de \LaTeX, pero esta plantilla dispone de comandos propios como por ejemplo el siguiente:

% \imagen{escudoInfor}{Autómata para una expresión vacía}{.5}



% \section{Listas de items}

% Existen tres posibilidades:

% \begin{itemize}
% 	\item primer item.
% 	\item segundo item.
% \end{itemize}

% \begin{enumerate}
% 	\item primer item.
% 	\item segundo item.
% \end{enumerate}

% \begin{description}
% 	\item[Primer item] más información sobre el primer item.
% 	\item[Segundo item] más información sobre el segundo item.
% \end{description}
	
% \begin{itemize}
% \item 
% \end{itemize}

% \section{Tablas}

% Igualmente se pueden usar los comandos específicos de \LaTeX o bien usar alguno de los comandos de la plantilla.

% \tablaSmall{Herramientas y tecnologías utilizadas en cada parte del proyecto}{l c c c c}{herramientasportipodeuso}
% { \multicolumn{1}{l}{Herramientas} & App AngularJS & API REST & BD & Memoria \\}{ 
% HTML5 & X & & &\\
% CSS3 & X & & &\\
% BOOTSTRAP & X & & &\\
% JavaScript & X & & &\\
% AngularJS & X & & &\\
% Bower & X & & &\\
% PHP & & X & &\\
% Karma + Jasmine & X & & &\\
% Slim framework & & X & &\\
% Idiorm & & X & &\\
% Composer & & X & &\\
% JSON & X & X & &\\
% PhpStorm & X & X & &\\
% MySQL & & & X &\\
% PhpMyAdmin & & & X &\\
% Git + BitBucket & X & X & X & X\\
% Mik\TeX{} & & & & X\\
% \TeX{}Maker & & & & X\\
% Astah & & & & X\\
% Balsamiq Mockups & X & & &\\
% VersionOne & X & X & X & X\\
% } 

\capitulo{4}{Técnicas y herramientas}
% Esta parte de la memoria tiene como objetivo presentar las técnicas metodológicas y las herramientas de desarrollo que se han utilizado para llevar a cabo el proyecto. Si se han estudiado diferentes alternativas de metodologías, herramientas, bibliotecas se puede hacer un resumen de los aspectos más destacados de cada alternativa, incluyendo comparativas entre las distintas opciones y una justificación de las elecciones realizadas. 
% No se pretende que este apartado se convierta en un capítulo de un libro dedicado a cada una de las alternativas, sino comentar los aspectos más destacados de cada opción, con un repaso somero a los fundamentos esenciales y referencias bibliográficas para que el lector pueda ampliar su conocimiento sobre el tema.

En este apartado se tratarán las distintas técnicas y herramientas utilizadas durante el desarrollo
del proyecto.

\section{Herramientas de gestión de proyectos}
Para la planificación del proyecto se ha buscado una herramienta que soporte
los elementos característicos de un desarrollo ágil como SCRUM permitiendo la creación
de distintas tareas y su gestión en el tiempo de desarrollo de la aplicación. 

Inicialmente se optó por Zenhub, aunque mientras se estaba desarrollando el proyecto
esta pasó a requerir una licencia de pago para su uso, por lo que se sustituyó esta
por Zube.

\subsection{Zenhub}
Zenhub ~\cite{zenhubZenhubProductivity} se trata de una herramienta de gestión de proyectos basada en las metodologías
ágiles, por lo que cuenta con funcionalidades que facilitan la creación de tareas, así como su gestión.
Además de crear tareas nos permite asignar a estas un \textit{Sprint}, puntos de poker (utilizados en SCRUM)
o poder definir aquellas tareas que lo requieran como \textit{Epics} para diferenciarlas de las otras.


Un punto importante para su elección inicial fue que está integrada con \textit{GitHub},
por lo que facilita la gestión de tareas al hacerla directamente desde el propio repositorio.
\subsection{Zube}
Una vez Zenhub pasó a ser una herramienta de pago, se optó por el uso de Zube ~\cite{zubeZubeAgile}. Esta cuenta con muchos
de los elementos que Zenhub también tiene. Permite definir tareas y gestionarlas mediante un tablero \textit{Kanban},
colocando las tareas en un área u otra en función de su estado (Backlog, Ready, In Progress,...).

\section{Lenguajes de programación}

Una elección bastante significativa para el proyecto será el lenguaje de programación empleado
para su desarrollo.
\subsection{Python}
Python~\cite{Python} es un lenguaje interpretado de alto nivel bastante utilizado en ámbitos como el \textit{Machine Learning}.
Además de lo anterior, cuenta con multitud de librerías de acceso abierto desarrolladas por los propios usuarios y/o la comunidad científica, lo cual es de gran utilidad en muchos aspectos.. En su elección también influye que es uno de los lenguajes que más hemos
utilizado durante la carrera.
\subsection{JavaScript}
Al igual que Python, JavaScript~\cite{JS} es un lenguaje interpretado, pero este es utilizado por los
navegadores para poder crear páginas web dinámicas. También cuenta con diversas librerías con funcionalidades
que permiten manejar el DOM, hacer distintas visualizaciones, entre otras. Este lenguaje será de gran importancia
en el apartado web del proyecto.

\section{Base de datos}

Dada la naturaleza del proyecto y los datos que se manejarán, una base de datos no relacional
puede dar ciertas facilidades en el manejo de estos. También cabe destacar que la información
tomará forma de grafo y se tratará usando técnicas propias de la ciencia de redes, por lo que 
estas necesidades tomarán gran importancia en la decisión de la base de datos.


\subsection{Neo4j}

Neo4j ~\cite{neo4j} es de una base de datos no relacional que pertenece a un nuevo paradigma en las bases de datos, la orientación a grafos, y está desarrollada con Java. Esta
es utilizada por grandes empresas como Intel, Adobe, AstraZeneca o incluso la NASA. También cabe destacar
su uso en el terreno de la ciencia de datos gracias a \textit{plugins} con utilidades propias de la ciencia de redes.

Cabe mencionar 
que cuenta con un lenguaje de consultas propio, \textit{Cypher}, con una sintaxis que 
facilita la obtención de información en un contexto de grafos.

Expuestos los hechos anteriores, Neo4j se parece la elección lógica para un proyecto de nuestras 
características.

\subsubsection{APOC (Awesome Procedures On Cypher)}
Como se ha mencionado antes, Neo4j cuenta con \textit{plugins} que facilitan algunas labores
en la base de datos. Entre ellos se encuentra APOC ~\cite{APOC}, que tiene tanto utilidades un tanto más generales,
así como algunas más específicas, como por ejemplo cargar datos directamente de peticiones a una API, que nos
será bastante útil en la carga de ubicaciones durante el desarrollo del proyecto.

\section{Librerías}
Como se ha mencionado anteriormente, se hará uso de diversas librerías para las labores del proyecto,
tanto de Python como de JavaScript.
\subsection{Driver de Neo4j}
Para el manejo de la base de datos de Neo4j a través de Python existen 3 distintos drivers ~\cite{neo4jpythonDrivers}:

\begin{description}
    \item [Driver Oficial] Se trata del driver oficial de Neo4j para Python. Cuenta con cursos para su uso y una extensa documentación.
    \item [Py2Neo] Es un driver creado por la comunidad de Neo4j con una una intefaz sencilla. Lamentablemente ya no recibe soporte para las últimas versiones de Neo4j.
    \item [Neomodel] Un driver alternativo al oficial que ofrece un OGM (Object Graph Mapper), similar a los ORMs de otras bases de datos; además de integración con el framework web \textit{Django}.
\end{description}

Dado que Neo4j es una herramienta nueva, que requiere de un proceso de aprendizaje, se ha optado
por el driver oficial ~\cite{neo4jdriver} por su simple uso y la gran cantidad de documentación existente. Se ha desestimado Py2Neo
puesto que ya no recibe soporte; y en el caso de Neomodel, las funcionalidades que trae no aportan una ventaja
respecto al driver oficial, puesto que en primer lugar no utilizaremos Django, sino Flask para la parte web del proyecto;
y en cuanto al OGM que provee, este no nos es necesario puesto que no requeriremos de las ventajas que este nos podría aportar.

\subsection{OSMPythonTools}
OSMPythonTools ~\cite{OSMP} se trata de una librería que facilita el acceso a la información de los servicios de \textit{Open Street Map}
mediante una interfaz simple. 

\subsection{Flask}
Flask~\cite{Flask} se trata de un framework de desarrollo web para Python. Está caracterizado por se ligero y flexible,
además de contar con librerías que expanden sus funcionalidades. Permite el uso de distintas bases de datos además de contar
con un motor de plantillas llamado \textit{Jinja2} para el renderizado de páginas web.

Se ha escogido este framework frente a otros debido a su simplicidad y fácil uso, además cubre todas
las necesidades del proyecto. Otro punto significativo en su elección es que ya se había trabajado con él
durante la carrera en la asignatura de \textit{Diseño y mantenimiento del Software}.

%\subsubsection{Flask JWT Extended}
%Es una librería que extiende las funcionalidades de Flask para poder utilizar JWT (JSON Web Token)
%para las labores de autenticación de usuarios y en la seguridad de la aplicación.
%\subsubsection{Flask WTForms}
%Extensión de Flask para la creación de formularios web mediante una interfaz simple, permitiendo
%la validación de estos desde el propio Python.

\subsection{Leaflet.js}
Se trata de una librería de código abierto de JavaScript que permite la visualización de mapas interactivos en páginas web ~\cite{leaflet}.
Para sus mapas puede utilizar información extraída de \textit{OpenStreetMap} para hacer las visualizaciones.
Nos servirá para mostrar las distintas ubicaciones que almacenemos en la base de datos en la web.

\subsection{Vis.js}
Es una librería de JavaScript cuyo objetivo es realizar visualizaciones interactivas de grafos en la web ~\cite{visjsVisjs}.
Permite personalizar las visualizaciones con distintas configuraciones. Con esta podremos mostrar
la información de la base de datos en forma de grafo en la forma que se haría desde el punto de vista de la ciencia
de redes.

\subsection{Bootstrap 5}
Se trata de un framework que se suele utilizar para la creación de interfaces web \textit{responsive} ~\cite{Bootstrap}.
Cuenta con una extensa documentación y ejemplos para su uso. Facilita el diseño de páginas web ofreciendo
componentes reutilizables pudiendo utilizarse directamente en las plantillas HTML.

\subsection{Scikit-Learn}
Scikit-Learn es una librería de código abierto de Python que provee herramientas y algoritmos de aprendizaje automático ampliamente utilizada en este ámbito~\cite{scikitlearnAbout}. En el caso de este trabajo se utilizará para crear los modelos de \textit{Random Forest} con los que se hará el sistema de recomendación.


\section{APIs}
Una parte importante de este proyecto será la información de las ubicaciones de las ciudades escogidas
para el proyecto. Para ello se obtendrán mediante APIs de información geográfica, en este caso se utilizará
OpenStreetMap para este propósito. 

OpenStreetMap es un proyecto abierto y colaborativo que cuenta con información geográfica
recopilada por los propios usuarios para la visualización de mapas, rutas de navegación y demás ~\cite{OSM}.

Para este proyecto se utilizarán las siguientes APIs de este proyecto. 

\subsection{Overpass}
Se trata de una de las APIs que componen el proyecto de OpenStreetMap, siendo esta solo de lectura de datos ~\cite{Overpass, openstreetmapOverpassOpenStreetMap}.
Está optimizada para las operaciones de lectura, contando con un lenguaje propio de consultas llamado \textit{Overpass QL}.
Además introduce algunas nuevas estructuras de datos con respecto a OpenStreetMap, como es el caso de las áreas, facilitando el acceso a la información en base a regiones.

\subsection{Nominatim}
Nominatim es un motor de búsqueda que forma parte de OpenStreetMap. Sus funcionalidades están centradas en la 
geolocalización en base a direcciones, nombres de ubicaciones y coordenadas ~\cite{nominatimNominatim}. Cuenta con una API propia
que será de utilidad para encontrar las áreas de las distintas ciudades con las que se trabajarán en el proyecto.

\section{Otros}
Además de todas las herramientas anteriormente mencionadas también se ha utilizado para facilitar el uso y despliegue de la aplicación.

\subsection{Docker}
Docker es una plataforma de código abierto utilizada para facilitar el despliegue de aplicaciones y servicios dentro de lo que se conoce como <<contenedores>>~\cite{dockerContainer}. Estos son capaces de contener todas las dependencias de la aplicación de forma que se pueda utilizar en cualquier equipo usando docker.  

\subsection{Heroku}
Heroku es una plataforma que permite desplegar aplicaciones web en la nube de forma gratuita~\cite{herokuWhatHeroku}. Cuenta con soporte para distintos lenguajes de programación, entre ellos Python. Nos permitirá desplegar la aplicación desarrollada en este proyecto de forma que su acceso sea más fácil.

\capitulo{5}{Aspectos relevantes del desarrollo del proyecto}

%Este apartado pretende recoger los aspectos más interesantes del desarrollo del proyecto, %comentados por los autores del mismo.
%Debe incluir desde la exposición del ciclo de vida utilizado, hasta los detalles de mayor %relevancia de las fases de análisis, diseño e implementación.
%Se busca que no sea una mera operación de copiar y pegar diagramas y extractos del código fuente, sino que realmente se justifiquen los caminos de solución que se han tomado, especialmente aquellos que no sean triviales.
%Puede ser el lugar más adecuado para documentar los aspectos más interesantes del diseño y de la %implementación, con un mayor hincapié en aspectos tales como el tipo de arquitectura elegido, los índices de las tablas de la base de datos, normalización y desnormalización, distribución en ficheros3, reglas de negocio dentro de las bases de datos (EDVHV GH GDWRV DFWLYDV), aspectos de desarrollo relacionados con el WWW...
%Este apartado, debe convertirse en el resumen de la experiencia práctica del proyecto, y por sí mismo justifica que la memoria se convierta en un documento útil, fuente de referencia para los autores, los tutores y futuros alumnos.


\section{Carga y almacenamiento de ubicaciones}

El primer paso del desarrollo del proyecto consiste en el almacenamiento de las ubicaciones obtenidas desde un servicio de geolocación como \textit{Open Street Map}. La elección de este sobre otras viene motivada debido principalmente a que se trata de un proyecto completamente abierto sin ninguna restricción monetaria sobre la extracción de datos, además es mantenido por la comunidad proporcionando documentación sobre el uso y la estructura de datos que se manejan.

Para la obtención de las ubicaciones se realizarían peticiones a la API de Overpass, siendo esta perteneciente al proyecto de Open Street Map. Esta es la que se suele utilizar para labores de lectura de datos.

\subsection{Carga inicial de ubicaciones}

Inicialmente se consideró el cargar las ubicaciones de distintas capitales de Europa, incluyéndose entre estas Madrid, París, Roma, Londres, Berlín y Amsterdam. Dado que Open Street Map cuenta con un sistema de etiquetado con formato \textit{clave-valor} sobre los elementos del su modelo de datos se determinar el incluir únicamente aquellos nodos con clave <<amenity>>. Esta se suele utilizar para cubrir diversos establecimientos públicos, servicios y negocios, por lo que se consideró adecuada para los objetivos buscados en el proyecto.

Para introducir las ubicaciones en la base de datos se optó inicialmente por usar la librería \textit{OSMPythonTools}, que provee una interfaz sencilla con la que hacer peticiones a la API. Debido a que el proceso conllevaba cierto tiempo, se empleó una alternativa con el plugin \textit{APOC} que permite cargar datos de la respuesta de una petición HTTP o un fichero JSON.

Una vez el proceso de carga se terminó, contábamos con un total de unos 280.000 nodos y 443 distintos valores de <<amenity>>. 

Al analizar los datos nos percatamos de que muchas <<amenities>> tenían fallos ortográficos o poseían un significado similar a otros. Este problema es debido a que las distintas claves de \textit{Open Street Map} no cuentan con un diccionario predefinido de posibles valores, sino que son los usuarios que crean las ubicaciones quienes escogen el valor de estas, estén registradas anteriormente o no.

Para resolver este problema se intentaron aplicar algoritmos de distancia de cadenas con el objetivo de unificar valores de <<amenity>>. Esta solución concluyó con que solo a partir de un 95\% de similitud se podrían juntar etiquetas con relativa precisión, aunque esta operación tendría que hacerse con supervisión para evitar posibles errores.

Otra solución que se buscó es utilizar la información de una página auxiliar de \textit{Open Street Map} que cuenta con estadísticas de los valores de las distintas etiquetas. Se intentó unificar las etiquetas a solo aquellas consideradas como <<oficiales>>, siendo estas las que cuentan con una página dentro de la wiki de \textit{Open Street Map}. Esta técnica se acabó desestimando debido a que muchas de las categorías con más aparición entre las ubicaciones cargadas no contaban con página en la wiki, siendo irrelevante este factor en su relevancia.

Adicionalmente se descubrió que gran cantidad de los nodos que estaban cargados en la base de datos pertenecían a mobiliario urbano (bancos, aparcamientos, papeleras, fuentes...), por lo que se consideró su eliminación debido a que no se tenían como tan importantes en comparación con el resto de categorías. Finalmente se desestimó para evitar el caso de un posible sesgo de los datos con su eliminación. Se terminó borrando aquellos nodos que contaban con un valor de \textit{amenity} auxiliar del modelo de datos de \textit{Open Street Map} (Yes, No, Fixme...) debido a que no contaban con ningún significado comercial o similar.


\subsubsection{Enlazado de nodos}

Mientras se encontraba una solución a los problemas enumerados anteriormente se decidió continuar con el enlazado de los nodos en la base de datos. Esto consistía en crear enlaces entre aquellos nodos que estuvieran en un radio de 100 metros. Para hacer esto se dotaron a los nodos con atributos tipo <<Point>>, creados a partir de la latitud y longitud de cada una de las ubicaciones, ya que estos eran valores obligatorios en los nodos de \textit{Open Street Map}. Una vez dotados a los nodos de estos atributos se crearon los enlaces entre los nodos con tal proximidad usando utilidades que el lenguaje de consultas Cypher provee para manejar distancias y coordenadas. Uno de los problemas de Neo4j es que posee es que solo permite crear enlaces unidireccionales. Si bien esto al principio resultaba problemático puesto que se creaban enlaces dobles entre cada par de nodos duplicando el número necesario de relaciones, al final se optó por crear un único enlace por par de nodos teniendo en consideración que habría que obviar la dirección en las consultas que se hagan a la base de datos sobre las ubicaciones, reduciendo así el número de aristas del grafo. Una vez este proceso se completó para todas las ciudades, se saldó el número total de enlaces de proximidad en 5.712.246. Dados los resultados anteriores se consideró que el número de conexiones era bastante superior al esperado y se desestimó el continuar con las ciudades actuales debido al esfuerzo computacional que conllevaría su procesamiento.

\section{Desarrollo Web}


\section{Sistema de Recomendación}
\capitulo{6}{Trabajos relacionados}

\section{\textit{Knowledge Transfer in Commercial Feature Extraction for the Retail Store Location Problem}}


Este artículo es la mayor inspiración detrás de este trabajo, puesto que tanto las técnicas que se han aplicado como la aproximación provienen de este. 

Al igual que en este artículo, se ha trabajado con ciudades de Castilla y León, aunque no con todas, pese a que no era la intención inicial esta elección de ciudades. 

Las principales diferencias con respecto lo realizado en este trabajo consisten en el origen de los datos y las categorías utilizadas. En el artículo los establecimientos comerciales son extraídos desde las Páginas Amarillas de 2017 y posteriormente geolocalizadas mediante APIs. En cuanto a categorías se utilizó 68 posibles valores definidos por \textit{North American Industry Classification for Small business} (NAICS).

Aplicaron los 3 métodos expuestos anteriormente, incluido el \textit{Rewiring} que no se pudo realizar en este trabajo, además de combinarlos mediante \textit{Random Forest}. Obtuvieron los siguientes resultados a nivel local:

\imagen{LocalPaper}{\textit{Mean Reciprocal Rank} por método y ciudad}{1}

En cuanto a la transferencia puesto que contaban con más ciudades, podían utilizar más combinaciones para probar su rendimiento.

\imagen{TransferPaper}{\textit{Mean Reciprocal Rank} al utilizar transferencia}{1}

En contraste con lo realizado con este trabajo, parece que la inclusión de nuevas ubicaciones pertenecientes a categorías no contempladas por el NAICS ayuda a obtener mejores resultados.

\section{\textit{Retail Store Location Selection Problem with Multiple Analytical Hierarchy Process of Decision Making an Application in Turkey}}

Este artículo es un ejemplo de otra aproximación al problema de selección de ubicación distinta a la ciencia de redes. En este emplean lo llamado \textit{Analytical Hierarchy Process} (AHP). Este consiste en definir distintos componentes tanto objetivos como subjetivos y hacer una decisión multicriterio en base a estos. Define 15 criterios agrupados en 5 categorías:


\begin{itemize}
		\item 			(M) Costes  \begin{itemize}
		\item \textbf{M1}: Coste del alquiler.
		\item \textbf{M2}: Coste del mobiliario.
		\item \textbf{M3}: Tiempos y condiciones de contratación.
	\end{itemize}
	\item  			(R) Competencia  \begin{itemize}
		\item \textbf{R1}: Poder de la competencia.
		\item \textbf{R2}: Número de competidores.
		\item \textbf{R3}: Distancia a la competencia.	
	\end{itemize} 
		\item 					(T) Densidad de tráfico  \begin{itemize}
			\item \textbf{T1}: Tráfico de vehículos.
			\item \textbf{T2}: Tráfico de viajeros.
		\end{itemize}
	\item 			(F) Característica físicas  \begin{itemize}
		\item \textbf{F1}: Tamaño de la tienda.
		\item \textbf{F2}: Aparcamientos.
		\item \textbf{F3}: Visibilidad.
	\end{itemize}
	\item 			(Y) Localización  \begin{itemize}
		\item \textbf{Y1}: Sobre calle principal.
		\item \textbf{Y2}: En centro comercial.
		\item \textbf{Y3}: Cercano a centros de negocio.
		\item \textbf{Y4}: Cercano a áreas residenciales y sociales.
	\end{itemize}
\end{itemize}

Si bien este artículo buscaba encontrar las mejores ubicaciones para un determinado negocio en 3 asentamientos, y por tanto, tomando una aproximación distinta a lo realizado en este proyecto, se puede apreciar la gran cantidad de información necesaria para poder estimar la mejor ubicación. Este problema no se presenta en nuestro proyecto, puesto que contando únicamente con las ubicaciones comerciales es capaz de inferir la adecuación de una localización a varias categorías comerciales, no solamente a una.




\capitulo{7}{Conclusiones y Líneas de trabajo futuras}

%Todo proyecto debe incluir las conclusiones que se derivan de su desarrollo. Éstas pueden ser de diferente índole, dependiendo de la tipología del proyecto, pero normalmente van a estar presentes un conjunto de conclusiones relacionadas con los resultados del proyecto y un conjunto de conclusiones técnicas. 
%Además, resulta muy útil realizar un informe crítico indicando cómo se puede mejorar el proyecto, o cómo se puede continuar trabajando en la línea del proyecto realizado. 

\section{Conclusiones}

En este proyecto se ha conseguido realizar un sistema de recomendación de categorías que puede ser de gran utilidad para todas aquellas personas que estén pensando en abrir un negocio, o simplemente para analizar la estructura comercial de las distintas ciudades.


Durante el desarrollo han surgido bastantes dificultades que no se preveían inicialmente que han obligado tomar distintas decisiones para llevar el proyecto a cabo. Muchas de estas están relacionadas con las herramientas utilizadas. Si bien el uso de estas nos ha llevado a problemas, cabe destacar que son las mejores opciones disponibles. Desarrollaré en los apartados sucesivos al respecto.

\subsection{Sobre la base de datos}

El uso de Neo4j vino principalmente motivado porque fue una de las herramientas propuestas para el proyecto. Tratándose de una base de datos orientada a grafos parecía ser apropiada para el desarrollo del trabajo, puesto que se han utilizado conceptos y estrategias propias de la ciencia de redes.

Desde el comienzo del proyecto se instó a usar lo máximo posible Neo4j para la mayoría de las etapas del proyecto con objetivo de aprovechar las funcionalidades que nos provee así cómo para justificar su uso de cara al trabajo.

Si bien es cierto que Neo4j hace bastante cómoda la obtención de datos de un grafo, no es tan flexible en otros aspectos. Esto se ha visto en los momentos en los que se buscaba aplicar los métodos de \textit{Rewiring} y \textit{Permutation}, así como el almacenamiento de los coeficientes que estos nos proporcionaban.

Cuando se requiere hacer una consulta más allá de lo más básico que se pueda hacer con \textit{Cypher}, el lenguaje de consultas de Neo4j, inevitablemente se tendrá que recurrir a procedimientos del plugin APOC. Esta característica hace que las consultas se vuelvan bastante tediosas y complejas, con varias llamadas a procedimientos de este plugin, dificultando su uso. Las capacidades que nos provee \textit{Cypher} nativo están bastante limitadas sin el uso de este plugin.

%https://github.com/neo4j/neo4j/issues/6282
Otro problema que plantea la base de datos es el modelo de datos que emplea. Neo4j no cuenta con la flexibilidad que otras bases de datos no relacionales tienen en este aspecto. Esto puede verse cuando se busca almacenar datos anidados en alguna de las entidades(nodos o relaciones), que no pueden almacenarse como una simple propiedad de estas sino que tienen que serializarse de un alguna forma (cómo se ha hecho en este trabajo) o recurrir a crear una nueva entidad relacionada que almacene dichos datos pese a que estos no tengan una relevancia suficiente como para considerase como nodos o relaciones, además de complicar las consultas que pudieran hacerse al respecto.

En cuanto a la realización de cálculos complejos también se ha visto que no es recomendable su uso más allá de las operaciones que estamos acostumbrados en otras bases de datos, puesto que tras intentar hacerlos se ha comprobado que son mucho más lentas de lo esperado. Aunque lo anterior puede que sea consecuencia de la complejidad de las consultas como se comentado anteriormente. Además en general estas operaciones son realmente difíciles de hacer para un usuario que no cuente con bastante experiencia anterior con esta base de datos.

Sobre conceptos propios de la ciencia de redes tampoco aporta mucha ayuda en ver las entidades de la base de datos cómo se haría desde esta disciplina, puesto que no aporta alguna representaciones utilizadas habitualmente como matrices o el no soporte a relaciones no dirigidas. Además resulta complicado trabajar con datos derivados de grafos para realizar operaciones o cálculos sobre el grafo, especialmente si estos constituyen una entidad propia como se ha hecho en este trabajo con las matrices de interacción y relacionados.

Habiendo expuesto lo anterior, parece que Neo4j no termina de comprometerse totalmente con ninguno de los conceptos en los que se basa, las bases de datos no relacionales y la ciencia de redes. No cuenta con la flexibilidad de otras bases de datos ni aporta algunos elementos propios de la ciencia de redes. Si alguien fuese a valorarse su uso, como opinión personal diría que debería verse como más próximo a las bases de datos no relacionales que a la ciencia de redes. Tras haber trabajado con ella parece más apropiada para otros contextos, como redes sociales y similares, en lugar de la temática de este trabajo. Además la amplia mayoría de funcionalidades que ofrece el \textit{plugin} \textit{Graph Data Science} están relacionadas con aspectos más convencionales de la ciencia de redes, como detección de comunidades, \textit{Random Walk}, medidas de centralidad y demás.

Pese a lo que se ha dicho, probablemente no exista una herramienta más adecuada para las labores realizadas en este proyecto que esta. También cabe mencionar que quizás se tenían unas expectativas no realistas con respecto a Neo4j. Si bien nos permite almacenar un grafo de forma persistente y operar con el directamente creo que no es competencia de una base de datos la realización de unas operaciones tan complejas como las que se han realizado, teniendo en su lugar que realizarse desde un lenguaje de programación que nos aporte una versatilidad que un lenguaje de consultas no es capaz de dar. Lo anterior sería lo ideal en caso de no contar con limitaciones como la capacidad de la memoria para alojar el grafo o similares, aunque tiene el impedimento de tener que construir el grafo mediante consultas, proceso que lleva tiempo, además de no poder operar directamente sobre el grafo en casos que se requiera hacer modificaciones o borrados ni obtener información tan fácilmente como se hace desde Neo4j mediante consultas de \textit{Cypher}.



\subsection{Sobre \textit{OpenStreetMap}}

\textit{OpenStreetMap} también se trataba de una de las herramientas propuestas. Si bien la obtención de datos no ha supuesto un problema de por sí ya que al tratarse de un proyecto abierto no ha habido barreras monetarias al respecto, y contar con un lenguaje de consultas propio para la obtención de datos; sí que lo ha habido sobre el contenido de las ubicaciones y su cantidad.

Uno de los problemas que nos ha presentado los datos son los valores de sus etiquetas de categorías. Cómo se ha mencionado en uno de los apartados del trabajo, no existe un diccionario de valores, sino que los usuarios son quienes los asignan. Esto conlleva a que existan gran cantidad de categorías, algunas con múltiples valores, faltas ortográficas y demás que han dificultado ciertas fases del proyecto.

Otro punto a destacar es la diferencia de número de ubicaciones en distintas ciudades, esto lejos de adecuarse a la realidad creo que viene determinado por la falta de cobertura de algunas zonas por parte del proyecto, además de la falta de puesta al día de la información de ciertas ubicaciones.

\textit{Overpass QL}, el lenguaje de consultas de esta API, si bien facilita la obtención de datos presenta problemas al obtener datos de ciertas áreas si existen más de una con ese mismo nombre. Tampoco pueden hacerse búsquedas de áreas dentro de áreas, cosa que es problemática al querer operar con determinadas zonas. En este caso planteó problemas al obtener datos de las ciudades finales del proyecto puesto que había más ciudades con ese nombre, por lo que se tuvo que recurrir al número de municipio para obtener sus nodos.

Expuesto lo anterior concluir que \textit{OpenStreetMap} supone una buena herramienta abierta para la obtención de datos, no sin sus inconvenientes, que probablemente si se hubiese utilizado una API de pago no se presentarían además de probablemente contar con una mejor calidad de datos.



\section{Líneas de trabajo futuras}

Debido a la gran cantidad de imprevistos que nos hemos encontrado durante proyecto hay apartados que han acabado con un desarrollo menor a lo esperado inicialmente. Además con los descubrimientos realizados podemos hacer algunas recomendaciones para proyectos similares a este o que busquen continuarlo.

Sobre las categorías de ubicaciones que se han utilizado, tras su empleo se ha llegado a la conclusión de que contar con un número elevado de categorías aumenta exponencialmente el esfuerzo computacional que conlleva la obtención de índices de calidad, así como las recomendaciones, además de contar con algunas que pueden ser redundantes o no relevantes. Lo más adecuado quizás sería emplear alguna forma de normalización de estas que nos permita transformarlas para tener un número más reducido de categorías aunque contemple todos los posibles valores, facilitando la obtención de métricas de calidad. Esto, sin embargo, debería hacerse con cuidado. Si se reducen las categorías a un número muy pequeño puede empeorar la calidad de los datos que disponemos, puesto que puede que no se capture adecuadamente la diversidad que hay entre categorías, agrupando varias con diferencias importantes entre sí, empeorando también las posibles recomendaciones que se pudieran hacer sobre ellas. Por el otro lado, si se reduce mínimamente las categorías se seguiría teniendo el mismo problema presentado anteriormente, siendo únicamente algo menos grave.

En cuanto a la calidad de los datos que nos ofrece \textit{OpenStreetMap}, como se ha observado en apartados anteriores, presentan problemas sobre los valores de categorías así como la irregularidad en el número de ubicaciones sobre algunas ciudades. Dado lo anterior, sería buena idea emplear APIs de pago que no se han podido utilizar como \textit{Google Maps}, a las que presuponemos una gran calidad de datos. Un punto a tener en consideración en la elección de estas sería el modelo de datos que emplean, especialmente el cómo estructuran las categorías comerciales, puesto que algunas emplean una lista de valores en lugar de un único valor; esto imposibilitaría la aplicación de los métodos de recomendación de este trabajo. De presentarse esto, podría aplicarse la normalización de categorías que se propone en el párrafo anterior para que el uso de los datos fuera factible.

La aplicación web no cuenta con muchas funcionalidades que se preveían inicialmente debido a los problemas presentados durante el desarrollo. Si bien están definidos roles, ninguno cuenta con permisos o utilidades especiales. Sería de bastante utilidad proveer a los administradores de una herramienta que les permita cargar datos de otras ciudades así como poder examinarlos y realizar los borrados y modificaciones que crean convenientes. Esto requeriría alojar la aplicación web en un sitio con la suficiente capacidad computacional como para aplicar los métodos realizados en este trabajo con objetivo de disponer de nuevas ciudades sobre las que realizar recomendaciones. En cuanto a los usuarios, se les podría dotar de la capacidad de guardar las ubicaciones sobre las que estén interesados, pudiendo acceder a ellas desde su perfil y realizar comparaciones sobre ellas.

En relación a los modelos utilizados para el sistema de recomendación solo se ha empleado \textit{Random Forest}, por lo que podría incluir nuevos modelos que el usuario pueda elegir con cuales realizar las recomendaciones.

El único método de obtención de métricas de calidad que ha faltado por implementar ha sido \textit{Rewiring}. Debido a los problemas que presenta este método no se utilizó en este proyecto, por lo que podría emplearse para desarrollos posteriores. Dada la complejidad que presenta y las dificultades de Neo4j para operar sobre grafos lo considero prácticamente irrealizable desde este, por lo que debería realizarse desde un lenguaje de programación.


\bibliographystyle{plain}
\bibliography{bibliografia}

\end{document}
