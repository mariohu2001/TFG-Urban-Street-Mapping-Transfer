\capitulo{1}{Introducción}

%Descripción del contenido del trabajo y del estrucutra de la memoria y del resto de materiales entregados.

El conocido como \textit{Retail Location Problem} consiste en un problema al que se enfrentan los negocios a la hora de elegir una ubicación donde abrir un nuevo establecimiento. Un correcto emplazamiento puede incrementar enormemente el rendimiento del negocio, proporcionando ventajas que la competencia difícilmente puede imitar. Esto puede suponer la diferencia entre el éxito o el fracaso 

\textit{Urban Street Mapping Transfer} se trata de un de un proyecto que busca ofrecer soluciones para este problema. La técnica que se utilizará para ello consiste en el análisis de la estructura comercial de cada ciudad mediante el uso de conceptos propios de la ciencia de redes así como de distintos métodos matemáticos y estadísticos que nos permitan estimar qué ubicaciones pueden ser más provechosas para un negocio. También se emplearán modelos de inteligencia artificial para integrar los resultados del análisis en un sistema de recomendación.

Además de lo anterior se utilizará \textit{knowledge transfer}, que permitirá utilizar datos de unas ciudades para poder realizar recomendaciones sobre otras. Todo lo anterior se realizará haciendo uso de una herramienta de gestión de bases de datos orientada a grafos, Neo4j.

Todo lo anterior supone un gran trabajo de investigación, tanto de aprendizaje de herramientas de nuevos paradigmas, de extracción de información y de aplicación de las diversas técnicas que se proponen en el trabajo. 
