\apendice{Anexo de sostenibilización curricular}

\section{Introducción}

Los objetivos de la sostenibilización curricular fueron definidos con el objetivo de crear unas directrices para los profesionales y estudiantes de educación superior concordes a la resolución de problemas globales y al desarrollo sostenible. Estos tienen el objetivo de crear profesionales capaces de utilizar sus habilidades para satisfacer necesidades sociales y ambientales~\cite{direcsost}.

La sostenibilidad es un concepto que abarca un deber moral relacionado con diversas cuestiones, como la protección del medio ambiente, la reducción de la pobreza, la igualdad de género, la promoción de la salud, los derechos humanos, la comprensión cultural, la paz, la producción y el consumo responsables, y el acceso equitativo a las tecnologías de la información y la comunicación (TIC), entre otros. Esta tiene como objetivo la contribución a la construcción de un mundo en donde todas las personas tengan sus necesidades básicas cubiertas, además de tener la capacidad de  adquirir una educación que les permita adoptar valores, comportamientos y estilos de vida coherentes con un futuro sostenible.

Estos objetivos han tenido efectos sobre los planes de estudio de todas las titulaciones universitarias. Estos deben incluir contenidos enfocados según los principios básicos~\cite{direcsost}:

\begin{enumerate}
	\item Principio ético: La universidad debe promover la educación ciudadana, priorizando la libertad y la protección de la vida, en armonía con el medio ambiente y con un enfoque en la equidad y los derechos de las generaciones actuales y futuras.
	\item Principio holístico: La universidad debe abordar los problemas sociales, económicos y ambientales desde una perspectiva integral e interdependiente, considerando aspectos ecológicos, sociales, económicos y éticos.
	\item Principio de complejidad: Se deben utilizar enfoques que contemplen a todo el sistema y de distintas disciplinas para comprender la complejidad de los problemas y su impacto en la sociedad.
	\item Principio de \textit{glocalización}: Se deben establecer conexiones entre los contenidos académicos y las realidades tanto locales como globales.
	\item Principio de transversalidad: La formación en sostenibilidad debe integrarse en todas las áreas del conocimiento, asignaturas y programas académicos, así como en la gestión, investigación y transferencia de la universidad.
	\item Principio de responsabilidad social universitaria: La universidad debe contribuir a la sostenibilidad de la comunidad, tanto en su funcionamiento interno como a través de proyectos de investigación y colaboración con otras entidades para abordar los problemas sociales, económicos y ambientales presentes en la sociedad.
\end{enumerate}

La inclusión de competencias relacionadas con la sostenibilidad en los planes de estudio de los estudios universitarios tiene como objetivos dotar al alumnado de 
la capacidad de comprender las distintas problemáticas sociales, económicas y ambientales presentes tanto en el ámbito local como global. Esto dotará al alumnado una ética coherente con los principios de la sostenibilidad, integrando esta de forma coherente en su toma de decisiones y en las distintas soluciones que proponga en su ámbito profesional~\cite{direcsost}.
%Este anexo incluirá una reflexión personal del alumnado sobre los aspectos de la sostenibilidad que se abordan en el trabajo.
%Se pueden incluir tantas subsecciones como sean necesarias con la intención de explicar las competencias de sostenibilidad adquiridas durante el alumnado y aplicadas al Trabajo de Fin de Grado.

%Más información en el documento de la CRUE \url{https://www.crue.org/wp-content/uploads/2020/02/Directrices_Sosteniblidad_Crue2012.pdf}.

%Este anexo tendrá una extensión comprendida entre 600 y 800 palabras.

\section{Aportación a la sostenibilidad}

Si bien los objetivos principales de este proyecto no orbitan directamente en torno al desarrollo sostenible, sí se podría decir que contribuye indirectamente. El fin último de este trabajo consiste en poder realizar recomendaciones con el objetivo de encontrar las ubicaciones más adecuadas para cierta categoría comercial o el establecimiento comercial más adecuado para una categoría. Esto, si bien en primera instancia puede parecer que no tiene relación con la sostenibilidad, indirectamente puede ayudar a conseguir estructuras comerciales más eficientes, y, por ende, más sostenibles, dentro del entorno urbano.

Si la localización comercial se realiza atendiendo al ecosistema de negocios vecinos más favorable, indudablemente mejorará su eficiencia, capacidad de generación de riqueza y sostenibilidad en el tiempo. Si los servicios y establecimientos se estableciesen de esta forma podría lograrse de mejor manera la satisfacción de las necesidades de la población, pudiendo suplir aquellas zonas donde existe una carencia de bienes o servicios, además de evitar aquellos negocios que no sean necesarios o que sean redundantes puesto que las necesidades que cubren ya están cubiertas, evitando de esta forma futuros fracasos comerciales y el desperdicio de inversiones.

Otro aspecto relevante es el reducido coste de un proyecto como este, en comparación con las alternativas comerciales que se ofrecen para este fin. En este sentido, nótese que el presente proyecto tiene vocación de ser escalado para poder utilizarlo no solo para un subgrupo de ciudades, sino para todas las ciudades de España y algunas ciudades europeas. Para ello, se han presentado propuestas a convocatorias competitivas de proyectos de investigación que van en esta línea. El objetivo es poder contribuir desde la investigación financiada con fondos públicos, a una mayor eficiencia económica de nuestros negocios, especialmente aquellos de tipo minorista
