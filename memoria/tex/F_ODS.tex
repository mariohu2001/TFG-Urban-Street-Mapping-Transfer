\apendice{Anexo de sostenibilización curricular}

\section{Introducción}

Los objetivos de la sostenibilización curricular fueron definidos con el objetivo de crear unas directrices para los profesionales y estudiantes de educación superior concordes a la resolución de problemas globales y al desarrollo sostenible. Estos tienen el objetivo de crear profesionales capaces de utilizar sus habilidades para satisfacer necesidades sociales y ambientales~\cite{direcsost}.
%Este anexo incluirá una reflexión personal del alumnado sobre los aspectos de la sostenibilidad que se abordan en el trabajo.
%Se pueden incluir tantas subsecciones como sean necesarias con la intención de explicar las competencias de sostenibilidad adquiridas durante el alumnado y aplicadas al Trabajo de Fin de Grado.

%Más información en el documento de la CRUE \url{https://www.crue.org/wp-content/uploads/2020/02/Directrices_Sosteniblidad_Crue2012.pdf}.

%Este anexo tendrá una extensión comprendida entre 600 y 800 palabras.

\section{Aportación a la sostenibilidad}

Si bien los objetivos principales de este proyecto no orbitan directamente en torno al desarrollo sostenible, sí se podría decir que contribuye indirectamente. El fin último de este trabajo consiste en poder realizar recomendaciones con el objetivo de encontrar las ubicaciones más adecuadas para cierta categoría comercial o el establecimiento comercial más adecuada para una categoría. Esto, si bien a primera instancia puede parecer que no tiene relación con la sostenibilidad, indirectamente puede ayudar a conseguir estructuras comerciales más eficientes dentro del ámbito urbano.

La adecuación de las ubicaciones y de los comercios puede que responda a las necesidades concretas de cada zona, por tanto, si la estructura comercial se organizase en base a esta podría conseguirse un modelo urbano más eficiente. Si los servicios y establecimientos se estableciesen de esta forma podría lograrse de mejor manera la satisfacción de las necesidades de la población, pudiendo suplir aquellas zonas donde existe una carencia de bienes o servicios, además de evitar aquellos negocios que no sean necesarios o que sean redundantes puesto que las necesidades que cubren ya están suplidas, evitando de esta forma futuros fracasos comerciales y el desperdicio material.

Otro punto a considerar es el relativo poco coste que ha sido necesario para obtener estas recomendaciones en comparación con métodos más convencionales en este ámbito.
