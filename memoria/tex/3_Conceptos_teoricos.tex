\capitulo{3}{Conceptos teóricos}

La cuestión que se busca abordar con este proyecto es el \textit{Retail Store Location Problem}. Este consiste
en la elección de la ubicación en la que colocar una nueva tienda de forma que esta le aporte el máximo beneficio posible.

\section{Retail Store Location Problem}

El conocido como <<Retail Store Location Problem>> consiste en encontrar
las mejores ubicaciones para un comercio de determinada categoría de forma 
que esta aporte ventajas competitivas con respecto a la competencia.

Originalmente este problema no se limitaba a comercios minoristas como se hará en este
trabajo, sino que también abarcaba ámbitos como la industria, teniendo en consideración
no solo la ubicación sino que también la distribución y recursos.


\section{Ciencia de Redes y Redes Complejas}
Uno de los puntos centrales teóricos del proyecto es el modelado del problema desde la
perspectiva de la Ciencia de Redes. Esta disciplina representa los objetos de estudio
como grafos, siendo estos compuestos por vértices y nodos. Esta representación permite analizar
la relación entre los objetos.

\subsection{Ciencia de redes}
La Ciencia de Redes se trata de una disciplina que se encarga del análisis de redes complejas. Esta representa los componentes como nodos o vértices, y las relaciones entre
estos como enlaces o aristas.

Este planteamiento puede aplicarse a distintos ámbitos y problemas, cómo las redes sociales,
control de transmisiones de enfermedades y demás. En este caso se usará para hacer recomendaciones de comercios en distintas ubicaciones.


\section{Técnicas}
%Construcción de la red de ubicaciones y comercios?
Para obtener las recomendaciones de ubicaciones contamos con 3 distintos métodos:
\begin{itemize}
	\item Jensen
	\item \textit{Permutation}
	\item \textit{Rewiring}
\end{itemize}

A continuación se hará una descripción de cada uno de estos métodos. Conviene hacer una
breve explicación de los términos utilizados.
\begin{align*}
	& T: \quad \text{Conjunto de todas las distintas ubicaciones} \\
	& A,B: \quad \text{Conjuntos de ubicaciones de determinadas categorías}\\
	& null\_model: \quad \text{Obtenido a partir de las simulaciones de Monte Carlo de \textit{Permutation} y  
		\textit{Rewiring}} \\
	& N_s(p,r): \quad \text{Tamaño del conjunto de ubicaciones de categoría \textit{s} en una proximidad de \textit{r} metros }
\end{align*}

\subsection{Jensen}
El método de Jensen consta de 2 distintos coeficientes, uno intracategoría (Una categoría consigo misma) e intercategoría (Una categoría con el resto).

\begin{equation*}
	M_\text{AA} = \frac{|T| - 1}{|A|(|A|-1)} \sum_{a \in A}\frac{N_A(a,r)}{N_T(a,r)}
\end{equation*}

\begin{equation*}
	M_\text{AB} = \frac{|T| - |A|}{|A||B|} \sum_{a \in A}\frac{N_B(a,r)}{N_T(a,r) - N_A(a,r)}
\end{equation*}

\subsection{Permutation}

\subsection{Rewiring}

\section{Métricas de Calidad}

\section{Transfer}
% En aquellos proyectos que necesiten para su comprensión y desarrollo de unos conceptos teóricos de una determinada materia o de un determinado dominio de conocimiento, debe existir un apartado que sintetice dichos conceptos.

% Algunos conceptos teóricos de \LaTeX \footnote{Créditos a los proyectos de Álvaro López Cantero: Configurador de Presupuestos y Roberto Izquierdo Amo: PLQuiz}.

% \section{Secciones}

% Las secciones se incluyen con el comando section.

% \subsection{Subsecciones}

% Además de secciones tenemos subsecciones.

% \subsubsection{Subsubsecciones}

% Y subsecciones. 


% \section{Referencias}

% Las referencias se incluyen en el texto usando cite \cite{wiki:latex}. Para citar webs, artículos o libros \cite{koza92}.


% \section{Imágenes}

% Se pueden incluir imágenes con los comandos standard de \LaTeX, pero esta plantilla dispone de comandos propios como por ejemplo el siguiente:

% \imagen{escudoInfor}{Autómata para una expresión vacía}{.5}



% \section{Listas de items}

% Existen tres posibilidades:

% \begin{itemize}
% 	\item primer item.
% 	\item segundo item.
% \end{itemize}

% \begin{enumerate}
% 	\item primer item.
% 	\item segundo item.
% \end{enumerate}

% \begin{description}
% 	\item[Primer item] más información sobre el primer item.
% 	\item[Segundo item] más información sobre el segundo item.
% \end{description}
	
% \begin{itemize}
% \item 
% \end{itemize}

% \section{Tablas}

% Igualmente se pueden usar los comandos específicos de \LaTeX o bien usar alguno de los comandos de la plantilla.

% \tablaSmall{Herramientas y tecnologías utilizadas en cada parte del proyecto}{l c c c c}{herramientasportipodeuso}
% { \multicolumn{1}{l}{Herramientas} & App AngularJS & API REST & BD & Memoria \\}{ 
% HTML5 & X & & &\\
% CSS3 & X & & &\\
% BOOTSTRAP & X & & &\\
% JavaScript & X & & &\\
% AngularJS & X & & &\\
% Bower & X & & &\\
% PHP & & X & &\\
% Karma + Jasmine & X & & &\\
% Slim framework & & X & &\\
% Idiorm & & X & &\\
% Composer & & X & &\\
% JSON & X & X & &\\
% PhpStorm & X & X & &\\
% MySQL & & & X &\\
% PhpMyAdmin & & & X &\\
% Git + BitBucket & X & X & X & X\\
% Mik\TeX{} & & & & X\\
% \TeX{}Maker & & & & X\\
% Astah & & & & X\\
% Balsamiq Mockups & X & & &\\
% VersionOne & X & X & X & X\\
% } 
