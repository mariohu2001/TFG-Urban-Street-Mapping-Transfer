\capitulo{2}{Objetivos del proyecto}

%Este apartado explica de forma precisa y concisa cuales son los objetivos que se persiguen con la realización del proyecto. Se puede distinguir entre los objetivos marcados por los requisitos del software a construir y los objetivos de carácter técnico que plantea a la hora de llevar a la práctica el proyecto.

Este proyecto tiene como objetivo principal la creación de recomendaciones de categorías comerciales dadas ubicaciones, de forma que se pueda estimar el mejor lugar para como localización de un determinado tipo de establecimiento comercial. Para ello se utilizarán distintos índices de calidad propuestos para este problema así como su uso combinado mediante un modelo de inteligencia artificial destinado a problemas de clasificación. Para facilitar el uso de la herramienta se diseñará una aplicación web sencilla que permita hacer las recomendaciones a un usuario.

Los objetivos para llevar a cabo el proyecto son los siguientes:

\begin{enumerate}
	\item Obtención de establecimientos comerciales con sus respectivas categorías de distintas ciudades mediante el uso de una API de geolocalización.
	\item Uso de una base de datos orientada a grafos donde almacenar las ubicaciones comerciales de forma que podamos operar sobre ella como se haría en una red.
	\item Análisis de la estructura comercial de cada ciudad así como de la interacción entre las distintas categorías.
	\item Uso y cálculo de distintos indices de calidad con objetivo de realizar recomendaciones.
	\item Uso de transferencia de información entre la estructura comercial de distintas ciudades.
	\item Creación de un sistema de recomendación basado en un modelo de inteligencia artificial.
	\item Creación de una aplicación web con objetivo de facilitar el uso de la herramienta.
\end{enumerate}

