\apendice{Documentación de usuario}

\section{Introducción}
Esta sección está centrará en el uso que se puede hacer de la aplicación por parte del usuario. Se explicarán las distintas funcionalidades y el uso esperado que se puede hacer de ellas.
\section{Requisitos de usuarios}
Dado que la herramienta desarrollada se trata de una aplicación web y no una aplicación de escritorio el usuario no estará sujeto a estrictos requisitos técnicos que puedan involucrar la compatibilidad de su equipo con la aplicación.

 A continuación se hará un pequeño listado de los requisitos esperados por parte de los usuarios:
 
 \begin{itemize}
 	\item \textbf{Dispositivo}: Ordenador o dispositivo móvil con capacidad de conexión a Internet.
 	\item \textbf{Conexión a Internet}: Dado que la herramienta se trata de una aplicación web es imprescindible contar con conexión a Internet.
 	\item \textbf{Navegador}: Se requiere un navegador actual compatible con HTML5 y CSS3.
 	\item \textbf{JavaScript}: Parte del funcionamiento de la aplicación del lado del cliente depende de JavaScript, por lo que este debe de estar habilitado en el navegador.
 	\item \textbf{Cuenta de usuario}: Algunas de las funcionalidades requieren de contar con una cuenta de usuario, además de haber iniciado sesión.
 \end{itemize}


\section{Instalación}

Salvo que se quiera realizar una ejecución a nivel local de la aplicación no es necesario realizar una instalación, bastando con simplemente acceder a la URL de la aplicación para poder hacer uso de ella.

En caso de que se quiera instalar localmente, se pueden seguir los pasos descritos en la sección de \hyperref[sec:compilación]{compilación, instalación y ejecución} en el manual del programador.
\section{Manual del usuario}


