\capitulo{4}{Técnicas y herramientas}
% Esta parte de la memoria tiene como objetivo presentar las técnicas metodológicas y las herramientas de desarrollo que se han utilizado para llevar a cabo el proyecto. Si se han estudiado diferentes alternativas de metodologías, herramientas, bibliotecas se puede hacer un resumen de los aspectos más destacados de cada alternativa, incluyendo comparativas entre las distintas opciones y una justificación de las elecciones realizadas. 
% No se pretende que este apartado se convierta en un capítulo de un libro dedicado a cada una de las alternativas, sino comentar los aspectos más destacados de cada opción, con un repaso somero a los fundamentos esenciales y referencias bibliográficas para que el lector pueda ampliar su conocimiento sobre el tema.

En este apartado se tratarán las distintas técnicas y herramientas utilizadas durante el desarrollo
del proyecto.

\section{Herramientas de gestión de proyectos}
Para la planificación del proyecto se ha buscado una herramienta que soporte
los elementos característicos de un desarrollo ágil como SCRUM permitiendo la creación
de distintas tareas y su gestión en el tiempo de desarrollo de la aplicación. 

Inicialmente se optó por Zenhub, aunque mientras se estaba desarrollando el proyecto
esta pasó a requerir una licencia de pago para su uso, por lo que se sustituyó esta
por Zube.

\subsection{Zenhub}
Zenhub se trata de una herramienta de gestión de proyectos basada en las metodologías
ágiles, por lo que cuenta con funcionalidades que facilitan la creación de tareas así como su gestión.
Además de crear tareas nos permite asignar a estas un \textit{Sprint}, puntos de poker (utilizados en SCRUM)
o poder definir aquellas tareas que lo requieran como \textit{Epics} para diferenciarlas de las otras.


Un punto importante para su elección inicial fue que esta cuenta con integración en \textit{GitHub},
por lo que facilita la gestión de tareas al hacerla directamente desde el propio repositorio.
\subsection{Zube}
Una vez Zenhub pasó a ser una herramienta de pago se optó por el uso de Zube. Esta cuenta con muchos
de los elementos que Zenhub también tiene. Permite definir tareas y gestionarlas mediante un tablero \textit{Kanban},
colocando las tareas en un area u otra en función de su estado (Backlog, Ready, In Progress,...)

\section{Lenguajes de programación}

Una elección bastante significativa para el proyecto será el lenguaje de programación empleado
para su desarrollo.
\subsection{Python}
Python se trata de un lenguaje interpretado de alto nivel bastante utilizado en ámbitos como el \textit{Machine Learning}.
Además de lo anterior cuenta con multitud de librerías de diversos propósitos desarrolladas por los propios usuarios, pudiendo
facilitar las labores a realizar en el proyecto. En su elección también influye que es uno de los lenguajes que más hemos
utilizado durante la carrera.
\subsection{JavaScript}
Al igual que Python, JavaScript se trata de un lenguaje interpretado, pero este es utilizado por los
navegadores para poder crear páginas web dinámicas. También cuenta con diversas librerías con funcionalidades
que permiten manejar el DOM, hacer distintas visualizaciones, entre otras. Este lenguaje será de gran importancia
en el apartado web del proyecto.

\section{Base de datos}

Dada la naturaleza del proyecto y los datos que se manejarán, una base de datos no relacional
puede dar ciertas facilidades en el manejo de estos. También cabe destacar que la información
tomará forma de grafo y se tratará usando técnicas propias de la ciencia de redes, por lo que 
estas necesidades tomarán gran importancia en la decisión de la base de datos.


\subsection{Neo4j}

Neo4j se trata de una base de datos no relacional de la familia de los grafos escrita en Java. Esta
es utilizada por grandes empresas como Intel, Adobe, AstraZeneca o incluso la NASA. También cabe destacar
su uso en el terreno de la ciencia de datos gracias a \textit{plugins} con utilidades propias de la ciencia de redes.

Cabe mencionar 
que cuenta con un lenguaje de consultas propio, \textit{Cypher}, con una sintaxis que 
facilita la obtención de información en un contexto de grafos.

Expuestos los hechos anteriores, Neo4j se trata de la elección lógica para un proyecto de nuestras 
características.

\subsubsection{APOC (Awesome Procedures On Cypher)}
Como se ha mencionado antes, Neo4j cuenta con \textit{plugins} que facilitan algunas labores
en la base de datos. Entre ellos se encuentra APOC, que tiene tanto utilidades un tanto más generales,
así como algunas más específicas, como por ejemplo cargar datos directamente de peticiones a una API, que nos
será bastante útil en la carga de ubicaciones durante el desarrollo del proyecto.
\section{Librerías}
Como se ha mencionado anteriormente, se hará uso de diversas librerías para las labores del proyecto,
tanto de Python como de JavaScript.
\subsection{Driver de Neo4j}
Para el manejo de la base de datos de Neo4j a través de Python existen 3 distintos drivers:

\begin{description}
    \item [Driver Oficial] Se trata del driver oficial de Neo4j para Python. Cuenta con cursos para su uso y una extensa documentación.
    \item [Py2Neo] Es un driver creado por la comunidad de Neo4j con una una intefaz sencilla. Lamentablemente ya no recibe soporte para las últimas versiones de Neo4j.
    \item [Neomodel] Un driver alternativo al oficial que ofrece un OGM (Object Graph Mapper), similar a los ORMs de otras bases de datos; además de integración con el framework web \textit{Django}.
\end{description}

Dado que Neo4j es una herramienta nueva, que requiere de un proceso de aprendizaje, se ha optado
por el driver oficial por su simple uso y la gran cantidad de documentación existente. Se ha desestimado Py2Neo
puesto que ya no recibe soporte; y en el caso de Neomodel, las funcionalidades que trae no aportan una ventaja
respecto al driver oficial, puesto que en primer lugar no utilizaremos Django, sino Flask para la parte web del proyecto;
y en cuanto al OGM que provee este no nos es necesario puesto que no requeriremos de las ventajas que este nos podría aportar.

\subsection{OSMPythonTools}
OSMPythonTools se trata de una librería que facilita el acceso a la información de los servicios de \textit{Open Street Map}
mediante una interfaz simple. 

\subsection{Flask}
Flask se trata de un framework de desarrollo web para Python. Está caracterizado por se ligero y flexible,
además de contar con librerías que expanden sus funcionalidades. Permite el uso de distintas bases de datos además de contar
con un motor de plantillas llamado \textit{Jinja2} para el renderizado de páginas web.

Se ha escogido este framework frente a otros debido a su simplicidad y fácil uso, además cubre todas
las necesidades del proyecto. Otro punto significativo en su elección es que ya se había trabajado con él
durante la carrera en la asignatura de \textit{Diseño y mantenimiento del Software}.

\subsubsection{Flask JWT Extended}
Es una librería que extiende las funcionalidades de Flask para poder utilizar JWT (JSON Web Token)
para las labores de autenticación de usuarios y en la seguridad de la aplicación.
\subsubsection{Flask WTForms}
Extensión de Flask para la creación de formularios web mediante una interfaz simple, permitiendo
la validación de estos desde el propio Python.

\subsection{Leaflet.js}
Se trata de una librería de código abierto de JavaScript que permite la visualización de mapas interactivos en páginas web.
Para sus mapas puede utilizar información extraída de \textit{OpenStreetMap} para hacer las visualizaciones.
Nos servirá para mostrar las distintas ubicaciones que almacenemos en la base de datos en la web.

\subsection{Vis.js}
Es una librería de JavaScript cuyo objetivo es realizar visualizaciones interactivas de grafos en la web.
Permite personalizar las visualizaciones con distintas configuraciones. Con esta podremos mostrar
la información de la base de datos en forma de grafo en la forma que se haría desde el punto de vista de la ciencia
de redes.

\subsection{Bootstrap 5}
Se trata de un framework que se suele utilizar para la creación de interfaces web \textit{responsive}.
Cuenta con una extensa documentación y ejemplos para su uso. Facilita el diseño de páginas web ofreciendo
componentes reutilizables pudiendo utilizarse directamente en las plantillas HTML.


\section{APIs}
Una parte importante de este proyecto será la información de las ubicaciones de las ciudades escogidas
para el proyecto. Para ello se obtendrán mediante APIs de información geográfica, en este caso se utilizará
OpenStreetMap para este propósito. 

OpenStreetMap se trata de un proyecto abierto y colaborativo que cuenta con información geográfica
recopilada por los propios usuarios para la visualización de mapas, rutas de navegación y demás.

Para este proyecto se utilizarán las siguientes APIs de este proyecto. 

\subsection{Overpass}
Se trata de una de las APIs que componen el proyecto de OpenStreetMap, siendo esta solo de lectura de datos.
Está optimizada para las operaciones de lectura, contando con un lenguaje propio de consultas llamado \textit{Overpass QL},
además de introducir algunas estructuras con respecto a OpenStreetMap, como es el caso de las areas, facilitando el acceso a la información.

\subsection{Nominatim}
Nominatim es un motor de búsqueda que forma parte de OpenStreetMap. Sus funcionalidades están centradas en la 
geolocalización en base a direcciones, nombres de ubicaciones y coordenadas. Cuenta con una API propia
que será de utilidad para encontrar las áreas de las distintas ciudades con las que se trabajarán en el proyecto.

